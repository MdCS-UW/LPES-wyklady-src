\PassOptionsToPackage{unicode=true}{hyperref}
\documentclass[12pt,aspectratio=169]{beamer}
\usepackage{fontspec,  xcolor, graphicx,  fancyvrb, fvextra}
\usetheme{boxes}
\setbeamertemplate{navigation symbols}{}
\setbeamertemplate{frametitle}[default][center]
\setsansfont{Latin Modern Sans}[Ligatures=NoCommon]
\begin{document}
\begin{frame}[fragile]

\frametitle{Definicja potencjału elektrycznego}

Potencjałem elektrycznym $V$ w dowolnym punkcie $P$ stałego pola elektrycznego nazywa się stosunek pracy $W$ wykonanej przez siłę elektryczną przy przenoszeniu ładunku $q$ z tego punktu do nieskończoności, do wartości tego ładunku:

$$V_{P}={\frac {W_{P\to \infty }}{q}}$$

Jednostką potencjału jest 1 V (wolt) równy 1 J/1 C (dżulowi na kulomb).


\vspace{2cm}\begin{flushright}
\footnotesize Źródło: Wikipedia, \url{https://pl.wikipedia.org/wiki/Potencjał_elektryczny}
\end{flushright}

\end{frame}
\end{document}
