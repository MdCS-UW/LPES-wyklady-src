\PassOptionsToPackage{unicode=true}{hyperref}
\documentclass[12pt,aspectratio=169]{beamer}
\usepackage{fontspec,  xcolor, graphicx,  fancyvrb, fvextra}
\usetheme{boxes}
\setbeamertemplate{navigation symbols}{}
\setbeamertemplate{frametitle}[default][center]
\setsansfont{Latin Modern Sans}[Ligatures=NoCommon]
\begin{document}
\begin{frame}[fragile]

\frametitle{Wyrażenia regularne}

\begin{Verbatim}
.      - dowolny znak
[a-z]  - znak z zakresu
[^a-z] - znak z poza zakresu
^      - początek napisu/linii
$      - koniec napisu/linii
\end{Verbatim}
\vspace{4pt}
\begin{Verbatim}
*      - dowolna ilość powtórzeń (także 0)
?      - 0 lub jedno powtórzenie
+      - jedno lub więcej powtórzeń
{n,m}  - od n do m powtórzeń
\end{Verbatim}
\vspace{4pt}
\begin{Verbatim}
()     - pod-wyrażenie (może być używane dla operatorów
         powtórzeń oraz dla referencji wstecznych)
|      - alternatywa: wystąpienie wyrażenia podanego po lewej stronie
         albo wyrażenia podanego prawej stronie
\end{Verbatim}

\end{frame}
\end{document}
