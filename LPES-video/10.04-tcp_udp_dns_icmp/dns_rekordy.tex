\PassOptionsToPackage{unicode=true}{hyperref}
\documentclass[12pt,aspectratio=169]{beamer}
\usepackage{fontspec,  xcolor, graphicx,  fancyvrb, fvextra}
\usetheme{boxes}
\setbeamertemplate{navigation symbols}{}
\setbeamertemplate{frametitle}[default][center]
\setsansfont{Latin Modern Sans}[Ligatures=NoCommon]
\begin{document}
\begin{frame}[fragile]

\frametitle{typy rekordów DNS}

\begin{itemize}
	\item \Verb@NS@    – informacja o serwerach obsługujących DNS danej domeny
	\item \Verb@A@     – mapowanie nazwy na adres IPv4
	\item \Verb@AAAA@  – mapowanie nazwy na adres IPv6
	\item \Verb@MX@    – informacja o serwerach obsługujących pocztę danej domeny
	\item \Verb@SRV@   – informacje o hoście świadczącym usługę w tej domenie {\footnotesize (usługa określana jest w nazwie domeny o którą pytamy)}
	\item \Verb@PTR@   – mapowanie adresów IP na nazwy domenowe, {\footnotesize realizowane w specjalnym drzewie \Verb@in-addr.arpa@ (dla IPv4) lub \Verb@ip6.arpa@ (IPv6),
	                     gdzie adres IP zapisywany jest w odwróconej kolejności po bajcie dla IPv4 lub cyfrze szesnastkowej dla IPv6}
	\item \Verb@TXT@   – informacje dodatkowe
	\item \Verb@SOA@   – informacje o strefie opisującej domenę
	\item \Verb@CNAME@ – alias na inną domenę {\footnotesize (domena którą aliasujemy nie może mieć innych wpisów, nawet SOA)}
\end{itemize}

\end{frame}
\end{document}
