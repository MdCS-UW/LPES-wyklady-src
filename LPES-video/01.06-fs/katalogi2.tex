\PassOptionsToPackage{unicode=true}{hyperref}
\documentclass[12pt,aspectratio=169]{beamer}
\usepackage{fontspec,  xcolor, graphicx,  fancyvrb, fvextra}
\usetheme{boxes}
\setbeamertemplate{navigation symbols}{}
\setbeamertemplate{frametitle}[default][center]
\setsansfont{Latin Modern Sans}[Ligatures=NoCommon]
\begin{document}
\begin{frame}[fragile]

\begin{itemize}
	\item \Verb{/etc}
	      konfiguracje ogólnosystemowe
	\item \Verb{/var}
	      dane programów i usług (takie jak kolejka poczty, harmonogramy zadań, bazy danych)
	\item \Verb{/home}
	      katalogi domowe użytkowników (często montowany z innego systemu plików, dlatego też root ma swój katalog domowy w \texttt{/root}, aby był dostępny nawet gdy takie montowanie nie doszło do skutku)
	\item \Verb{/tmp}
	      pliki tymczasowe (typowo czyszczony przy starcie systemu);
	\item \Verb{/run}
		dane tymczasowych działających usług, takie jak numery pid, blokady, itp. (specyficzny dla Linuxa)
\end{itemize}

\end{frame}
\end{document}
