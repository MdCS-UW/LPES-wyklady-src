\PassOptionsToPackage{unicode=true}{hyperref}
\documentclass[11pt,aspectratio=169]{beamer}
\usepackage{fontspec,  xcolor, graphicx,  fancyvrb, fvextra}
\usetheme{boxes}
\setbeamertemplate{navigation symbols}{}
\setbeamertemplate{frametitle}[default][center]
\setsansfont{Latin Modern Sans}[Ligatures=NoCommon]
\begin{document}
\begin{frame}[fragile]

\frametitle{Warstwy modelu OSI}

\begin{enumerate}
	\item \textbf{fizyczna (L1)} – aspekty związane z fizycznym przesyłem sygnału takie jak częstotliwości radiowe, poziomy napięć, etc.;\\
		określa sposób transmisji kolejnych bajtów
	\item \textbf{łącza danych (L2)} – aspekty związane z formatem ramki, protokoły zasad dostępu do medium transmisyjnego, itd.;\\
		określa sposób transmisji porcji danych pomiędzy hostami w jednej sieci
	\item \textbf{sieciowa (L3)} – aspekty związane z formatem pakietu, adresacja i zasady routingu umożliwiające zapewnienie łączności pomiędzy różnymi sieciami;\\
		określa sposoby transmisji porcji danych pomiędzy sieciami
	\item \textbf{transportowa (L4)} – podział strumienia na porcje informacji, kontrola nad poprawnością transmisji, adresację usług w ramach hosta
	\item \textbf{sesji}
	\item \textbf{prezentacji}
	\item \textbf{aplikacji}
\end{enumerate}

\end{frame}
\end{document}
