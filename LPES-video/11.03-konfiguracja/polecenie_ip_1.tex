\PassOptionsToPackage{unicode=true}{hyperref}
\documentclass[11pt,aspectratio=169]{beamer}
\usepackage{fontspec,  xcolor, graphicx,  fancyvrb, fvextra}
\usetheme{boxes}
\setbeamertemplate{navigation symbols}{}
\setbeamertemplate{frametitle}[default][center]
\setsansfont{Latin Modern Sans}[Ligatures=NoCommon]
\begin{document}
\begin{frame}[fragile]

\begin{itemize}
	\item \Verb{ip addr ...} – adresy IP
		\begin{itemize}
			\item \Verb{ip addr} – wypisuje obecną konfigurację adresów i informacje o stanie interfejsu
			                 (\Verb{UP}/\Verb{DOWN} – interfejs włączony/wyłączony,
			                  \Verb{LOWER_UP}/\Verb{LOWER_DOWN} – link warstwy niższej na interfejsie / jego brak)
			\item \Verb{ip addr add ADDRESS dev INTERFACE} – dodaje adres \Verb{ADDRESS} do interfejsu \Verb{INTERFACE}
			\item \Verb{ip addr del ADDRESS dev INTERFACE} – usuwa adres \Verb{ADDRESS} z interfejsu \Verb{INTERFACE}
		\end{itemize}
	\vspace{7pt}
	\item \Verb{ip route ...} – routing IP
		\begin{itemize}
			\item \Verb{ip [-6] route} – wyświetlanie informacji na temat tras routingowych dla IPv4 (gdy wywołany bez opcji \Verb{-6}) / IPv6 (gdy wywołany z opcją \Verb{-6})
			\item \Verb{ip route add NETWORK via GATEWAY dev INTERFACE} – dodanie trasy routingowej do sieci \Verb{NETWORK} poprzez router o adresie \Verb{GATEWAY} na interfejsie \Verb{INTERFACE}
			\item \Verb{ip route del NETWORK via GATEWAY dev INTERFACE} – usunięcie trasy routingowej do sieci \Verb{NETWORK} ...
		\end{itemize}
\end{itemize}

\end{frame}
\end{document}
