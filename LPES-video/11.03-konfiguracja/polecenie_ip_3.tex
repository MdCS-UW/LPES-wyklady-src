\PassOptionsToPackage{unicode=true}{hyperref}
\documentclass[11pt,aspectratio=169]{beamer}
\usepackage{fontspec,  xcolor, graphicx,  fancyvrb, fvextra}
\usetheme{boxes}
\setbeamertemplate{navigation symbols}{}
\setbeamertemplate{frametitle}[default][center]
\setsansfont{Latin Modern Sans}[Ligatures=NoCommon]
\begin{document}
\begin{frame}[fragile]

\begin{itemize}
	\item konfiguracja BRIDGE (programowego switcha)
		\begin{itemize}
			\item \Verb{ip link add IFACE type bridge} – dodanie interfejsu bridge'owego o nazwie IFACE
			\item \Verb{ip link set SLAVE master IFACE}  – włączenie interfejsu SLAVE w skład bridge'owego IFACE
			\item \Verb{ip link set SLAVE nomaster} – wyłaczenie interfejsu SLAVE z bridge'a
			\item \Verb{ip link show master IFACE} – wyświetlanie portów składowych bridge'a o nazwie IFACE
			\item przydatne może być także polecenie \Verb{bridge}
		\end{itemize}
	\vspace{7pt}
	\item konfiguracja BONDów (interfejsów agregujących inne w grupę celem zwiększenia prędkości lub niezawodności)
		\begin{itemize}
			\item \Verb{ip link add IFACE type bond} – dodanie interfejsu bondingowego o nazwie IFACE
			\item \Verb{ip link set SLAVE master IFACE}  – włączenie interfejsu SLAVE w skład bondingu IFACE
			\item \Verb{ip link set SLAVE nomaster} - wyłaczenie interfejsu SLAVE z bondingu
			\item \Verb{ip link show master IFACE} – wyświetlanie portów składowych interfejsu bondingowego o nazwie IFACE
		\end{itemize}
\end{itemize}

\end{frame}
\end{document}
