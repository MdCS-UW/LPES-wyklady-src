\renewcommand{\zaawansowane}[1]{%
	\ifnumcomp{#1}{<}{5}  {} {%
	\ifnumcomp{#1}{<}{15} {} {%
	\ifnumcomp{#1}{<}{25} {\Symbola 🤔} {\Symbola 🧐}%
	}}%
}

\makeatletter\hypersetup{
	pdftitle = {\@title}, pdfauthor = {\@author}
}\makeatother

\makeatletter\let\percentchar\@percentchar\makeatother
\newcommand{\draftDate}{
	\directlua{
		if not os.getenv("LPES_FINAL") then
			if os.getenv("LPES_DRAFT_DATE") then
				tex.sprint( " [draft " .. os.getenv("LPES_DRAFT_DATE") .. "]" )
			else
				tex.sprint( " [draft " .. os.date("\percentchar F") .. "]" )
			end
		end
	}
}
\makeatletter
\let\oldDate\@date
\date {\oldDate \color{red}{\textbf{\draftDate}}}
\makeatother

\newcommand{\baseURLtoLPES}{http://ciekawi.icm.edu.pl/lpes}


\NewDocumentCommand{\Zadania}{o m o}{
	\section{Zadania}
	\IfValueT{#1}{\input{#1}}
	\renewcommand*{\do}[1]{\input{##1}}
	\docsvlist{#2}
	
	\IfValueT{#3}{
		\vspace{1cm}
		\section{Zadania praktyczne}
		
		Zadania te, dokładnie w takiej samej formie, będziemy realizować wspólnie w ramach laboratorium, więc nie musisz ich robić samemu.
		Zamieszczamy je jednak z wyprzedzeniem, abyś wiedział(a) co cię czeka i upewnił(a) się że masz wszystkie potrzebne elementy pod ręką.
		
		\renewcommand*{\do}[1]{\dbEntryInsert{##1}{praktyczne}}
		\docsvlist{#3}
	}
	\renewcommand{\insertZadanie}[3]{\subsubsection*{Rozwiązanie zadania \ref{##2}} \dbEntryInsert{##1}{##2-rozwiazanie}}
	\newcommand{\noExtraInfoMode}{TRUE}
	\input{booklets-sections/rozwiazania-intro.tex}
	\begin{RozwiazanieBox}
	\renewcommand*{\do}[1]{\input{##1}}
	\docsvlist{#2}
	\end{RozwiazanieBox}
	\let\noExtraInfoMode\undefined
}
