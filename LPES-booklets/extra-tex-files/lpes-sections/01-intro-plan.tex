\section{Plan kursu}

Kurs składa się z następujących bloków tematycznych:

\vspace{11pt}\begin{easylist}[enumerate]
	& Linux: wprowadzenie
	& Python: wprowadzenie, pętle i funkcje, napisy, wyrażenia regularne
	& Python: listy, słowniki, obiekty, pliki, wyjątki, biblioteki
	& Linux: operacje na plikach tekstowych, awk, praca zdalna, użytkownicy, procesy
	& Linux: programowanie w bashu, system operacyjny, usługi
	& C i C++: podstawy C, adres zmiennej (wskaźniki), współczesny C++
	& Elektronika: podstawy, elementy bierne, dioda, tranzystor
	& Elektronika cyfrowa: bramki i przerzutniki, transmisja, magistrale, układy programowalne
	& Sieci komputerowe: protokoły, adresy, routing, usługi, standardy
	& Sieci komputerowe: ethernet, konfiguracja programowanie usług   \itemLabel{ostatniPelnyWyklad}
	& Wprowadzenie do programowania mikrokontrolerów STM32   \itemLabel{pierwszyLab}
\end{easylist}


\begin{comment}
\subsubsection{Kurs uzupełniający}
Po zakończeniu podstawowej części kursu przewidziano także kilka labolatoriów, które odbędą się w ramach edycji „bis”.
Będą to równierz jako 3.5 godzinne spotkania, jednak w całości poświęcone zadaniom praktycznym, a warunkiem nazpisu na nie będzie ukończenie podstawowego kursu.
Planowane są następujące spotkania:
\vspace{5pt}\begin{easylist}[itemize]
	& Konfiguracja sieci – konfiguracja, vlan'y i routing \teacher{
	  zajęcia na ogół dobrze oceniane i IMHO pożyteczne, ale trudne do realizowania na większą skalę
	  – wymaga stanowisk sprzętowych (3 komputery z dostępem do root'a switch zarządzany per stanowisko)
	  i prowadzącego biegłego w tych zagadnieniach, aby mógł rozwiązywać nieprzewidziane problemy}
	
	& Komputery jednopłytkowe (typu Raspbbery Pi, z systemem Linux) i mikrokontrolery – komunikacja z podzespołami elektronicznymi (UART, I2C), podstawy administracji systemem
	
	& Instalacja Linuxa i konfiguracja systemu oraz usług
	
	& Współczesny C++, łączenie Pythona z C++ i inne biblioteki (SQL, GUI, ...)
\end{easylist}
\end{comment}
