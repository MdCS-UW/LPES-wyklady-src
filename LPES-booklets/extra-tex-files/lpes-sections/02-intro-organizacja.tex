\section{Organizacja zajęć}

Na bloki 1–\ref{ostatniPelnyWyklad} składa się:
\vspace{3pt}\begin{easylist}[itemize]
	& wykład wideo złożony z 4 – 6 filmów o łącznym czasie trwania 45 – 65 minut
	& skrypt wykładowy zawierający spisane najważniejsze informacje z wykładu, przykładowe kody i polecenia oraz zadania (wraz z rozwiązaniami)
	& ćwiczenia w ramach których będziemy dyskutować o zagadnieniach poruszonych na wykładzie, omówimy rozwiązania zadań ze skryptu wykładowego i będziemy rozwiązywać kolejne
	& pracy domowej
\end{easylist}\vspace{7pt}

\noindent
Blok \ref{pierwszyLab} będzie wyglądać podobnie, tyle że jego część wykładowa będzie znacznie krótsza i być może zostanie nawet dołączona do któregoś z wcześniejszych wykładów oraz nie będzie zawierać on pracy domowej.
Planujemy także dodatkowy blok ćwiczeniowy (bez typowej części wykładowej) wprowadzający do pracowni elektronicznej on-line.

\vspace{7pt}\noindent
Zajęcia tej edycji realizowane będą w systemie on-line, więc będą wymagały większego zaangażowania i samodyscypliny z Waszej strony.
Oczekujemy że przed ćwiczeniami zapoznacie się z odpowiednimi materiałami wykładowymi, w szczególności wykładem wideo.
Możecie to zrobić w dowolnym miejscu i czasie od jego premiery, ale \textbf{wymagamy} abyście obejrzeli cały wykład \textbf{przed ćwiczeniami}.
Zachęcamy też do zapoznania się ze skryptem wykładowym i próbą samodzielnego rozwiązania zamieszczonych w nim zadań.

\subsection{Terminy zajęć}

\begin{easylist}[itemize]
	& \textbf{Ćwiczenia} będą odbywać się od 17.00 do 19.15 (z przerwą na zjedzenie ciastka) w każdy wtorek od 9 marca do 8 czerwca (z wyjątkiem 6 kwietnia).
	  Udział w ćwiczeniach jest \textbf{obowiązkowy}.
	& Premiery \textbf{wykładów} będą odbywać się w czwartki poprzedzające ćwiczenia o 17.00 na platformie YouTube.
	  W trakcie premiery zapraszamy do zadawania pytań związanych z aktualnym tematem na naszym serwerze Discord.
	  Oczywiście wykłady można będzie obejrzeć także w dowolnym czasie po premierze (będą one dostępne na naszym kanale nawet po zakończeniu kursu).
	  Udział w premierze wykładu jest nieobowiązkowy, ale zapoznanie się z wykładem wideo przed ćwiczeniami jest \textbf{obowiązkowe}.
	  
	& Na \textbf{prace domowe} będziemy czekać do 23$^{59}$ w sobotę po ćwiczeniach na których została zadana.
	  Prace domowe są \textbf{nieobowiązkowe}, ale dają punkty w rywalizacji indywidualnej i całych szkół.
\end{easylist}
