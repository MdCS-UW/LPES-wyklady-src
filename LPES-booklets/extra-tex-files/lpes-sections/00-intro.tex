\section{O zajęciach}

Kurs \textit{„Linux i Python w Elektronicznej Sieci”} jest intensywnym wprowadzeniem w najważniejsze zagadnienia związane z systemami typu Unix, programowaniem, sieciami komputerowymi oraz podstawami elektroniki, która stoi za działaniem komputerów, sieci komputerowych i dużej części współczesnego świata. W ramach kursu zapoznasz się:
\begin{itemize}
	\item z kilkoma z pośród najistotniejszych języków programowania (w tym z Pythonem i C),
	\item pracą w systemach typu Unix/Linux oraz ich działaniem,
	\item działaniem, budową i wykorzystywaniem sieci komputerowych,
	\item podstawami elektroniki analogowej i cyfrowej oraz programowania mikrokontrolerów.
\end{itemize}
Kurs przeznaczony jest dla osób zainteresowanych tą tematyką i posiadających elementarną wiedzą związaną z tymi zagadnieniami (podstawy programistyczne, wiedzę z fizyki z zakresu elektryczności, itp).
Zagadnienia na kursie w miarę możliwości omawiane będą od podstaw, jednak ze względu na intensywność kursu omówienie podstaw należy traktować raczej jedynie jako przypomnienie.
Celem kursu jest ułatwienie dalszego zgłębiania tajników szeroko rozumianej informatyki i elektroniki poprzez przekazanie gruntownych podstaw oraz ich uporządkowanie i usystematyzowanie.
Staramy się przekazywać praktyczną wiedzę i w taki sposób podchodzić do omawianych zagadnień.
Jednak chcielibyśmy abyś po ukończeniu kursu nie tylko potrafił(a) samodzielnie rozwiązywać problemy związane z omawianymi zagadnieniami („coś zrobić”),
ale także abyś rozumiał(a) „jak to działa?” i był(a) wstanie samodzielnie zgłębiać wybrane zagadnienia. Zatem nie unikniemy niezbędnej teorii.

Ze względu na ograniczony czas trwania kursu, niektóre duże tematy będą jedynie wspomniane lub pokazane na pojedynczych przykładach.
Mamy nadzieję, iż zainteresują one przynajmniej niektórych z Was i będą inspiracją do samodzielnego ich poznania w szerszym zakresie.
Prace domowe są nie obowiązkowe, ale punktowane.
Duża ilość praktyki jest bardzo ważna, zatem zachęcamy do ich odrabiania oraz samodzielnego eksperymentowania z przykładami / kodami omawianymi na zajęciach.

\begin{teacherOnly}
Kurs oparty jest o środowisko GNU/Linux. Wybór ten podyktowany jest kilkoma czynnikami:
\begin{itemize}
\item bo jest bardzo popularnym środowiskiem „unixowatym”
\item bo jest wygodnym środowiskiem dla programisty, informatyka, itp
\item bo wiele rzeczy jest dużo prostszych niż gdzie indziej, np. aby zainstalować Pythona 3 wystarczy jedno polecenie (w pochodnych Debiana \Verb{sudo apt-get install python3})
\item bo dostęp do kodu systemu, używanych bibliotek i narzędzi potrafi się przydać
\end{itemize}
\end{teacherOnly}

