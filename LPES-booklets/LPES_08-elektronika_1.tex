% Copyright (c) 2020-2021 Matematyka dla Ciekawych Świata (http://ciekawi.icm.edu.pl/)
% Copyright (c) 2020-2021 Robert Ryszard Paciorek <rrp@opcode.eu.org>
% 
% MIT License
% 
% Permission is hereby granted, free of charge, to any person obtaining a copy
% of this software and associated documentation files (the "Software"), to deal
% in the Software without restriction, including without limitation the rights
% to use, copy, modify, merge, publish, distribute, sublicense, and/or sell
% copies of the Software, and to permit persons to whom the Software is
% furnished to do so, subject to the following conditions:
% 
% The above copyright notice and this permission notice shall be included in all
% copies or substantial portions of the Software.
% 
% THE SOFTWARE IS PROVIDED "AS IS", WITHOUT WARRANTY OF ANY KIND, EXPRESS OR
% IMPLIED, INCLUDING BUT NOT LIMITED TO THE WARRANTIES OF MERCHANTABILITY,
% FITNESS FOR A PARTICULAR PURPOSE AND NONINFRINGEMENT. IN NO EVENT SHALL THE
% AUTHORS OR COPYRIGHT HOLDERS BE LIABLE FOR ANY CLAIM, DAMAGES OR OTHER
% LIABILITY, WHETHER IN AN ACTION OF CONTRACT, TORT OR OTHERWISE, ARISING FROM,
% OUT OF OR IN CONNECTION WITH THE SOFTWARE OR THE USE OR OTHER DEALINGS IN THE
% SOFTWARE.

\documentclass{pdfBooklets}

\title{Linux i Python w Elektronicznej Sieci \#08:\\ Wprowadzenie do elektroniki}
\author{%
	Projekt ,,Matematyka dla Ciekawych Świata'',\\
	Robert Ryszard Paciorek\\\normalsize\ttfamily <rrp@opcode.eu.org>
}
\date  {2021-04-29}

\renewcommand{\zaawansowane}[1]{%
	\ifnumcomp{#1}{<}{5}  {} {%
	\ifnumcomp{#1}{<}{15} {} {%
	\ifnumcomp{#1}{<}{25} {\Symbola 🤔} {\Symbola 🧐}%
	}}%
}

\makeatletter\hypersetup{
	pdftitle = {\@title}, pdfauthor = {\@author}
}\makeatother

\makeatletter\let\percentchar\@percentchar\makeatother
\newcommand{\draftDate}{
	\directlua{
		if not os.getenv("LPES_FINAL") then
			if os.getenv("LPES_DRAFT_DATE") then
				tex.sprint( " [draft " .. os.getenv("LPES_DRAFT_DATE") .. "]" )
			else
				tex.sprint( " [draft " .. os.date("\percentchar F") .. "]" )
			end
		end
	}
}
\makeatletter
\let\oldDate\@date
\date {\oldDate \color{red}{\textbf{\draftDate}}}
\makeatother

\newcommand{\baseURLtoLPES}{http://ciekawi.icm.edu.pl/lpes}


\NewDocumentCommand{\Zadania}{o m o}{
	\section{Zadania}
	\IfValueT{#1}{\input{#1}}
	\renewcommand*{\do}[1]{\input{##1}}
	\docsvlist{#2}
	
	\IfValueT{#3}{
		\vspace{1cm}
		\section{Zadania praktyczne}
		
		Zadania te, dokładnie w takiej samej formie, będziemy realizować wspólnie w ramach laboratorium, więc nie musisz ich robić samemu.
		Zamieszczamy je jednak z wyprzedzeniem, abyś wiedział(a) co cię czeka i upewnił(a) się że masz wszystkie potrzebne elementy pod ręką.
		
		\renewcommand*{\do}[1]{\dbEntryInsert{##1}{praktyczne}}
		\docsvlist{#3}
	}
	\renewcommand{\insertZadanie}[3]{\subsubsection*{Rozwiązanie zadania \ref{##2}} \dbEntryInsert{##1}{##2-rozwiazanie}}
	\newcommand{\noExtraInfoMode}{TRUE}
	\input{booklets-sections/rozwiazania-intro.tex}
	\begin{RozwiazanieBox}
	\renewcommand*{\do}[1]{\input{##1}}
	\docsvlist{#2}
	\end{RozwiazanieBox}
	\let\noExtraInfoMode\undefined
}


\begin{document}

\maketitle

\input{booklets-sections/electronics/10-podstawy.tex}
\input{booklets-sections/electronics/11-bierne.tex}
\input{booklets-sections/electronics/12-dioda.tex}
\clearpage
\input{booklets-sections/electronics/13-tranzystor.tex}

\section{Wykład wideo}
\input{booklets-sections/electronics/wykład-video-1.tex}

\ZadaniaRozwiazaniaAuto

\section{Zadania praktyczne}
	\newcommand{\ladowanieKondensatoraWartosci}{Wskazówka: możesz zacząć od R1 = R2 = 22kΩ.}
	\insertZadanie{booklets-sections/electronics/zadania-10-12-podstawy.tex}{zbuduj_ladowanie_kondensatora}{}
	\insertZadanie{booklets-sections/electronics/zadania-10-12-podstawy.tex}{zbuduj_spadek_napiecia_na_led}{}
	\insertZadanie{booklets-sections/electronics/zadania-10-12-podstawy.tex}{zbuduj_stabilizator_zener1}{}
	\insertZadanie{booklets-sections/electronics/zadania-13-tranzystor.tex}{zbuduj_klucz_npn1}{}
	\insertZadanie{booklets-sections/electronics/zadania-13-tranzystor.tex}{zbuduj_klucz_npn2}{}
	\insertZadanie{booklets-sections/electronics/zadania-13-tranzystor.tex}{zbuduj_stabilizator_zener2}{}
	\insertZadanie{booklets-sections/electronics/zadania-13-tranzystor.tex}{wzmacniacz_operacyjny}{}

\copyrightFooter{
	© Matematyka dla Ciekawych Świata, 2017-2021.\\
	© Robert Ryszard Paciorek <rrp@opcode.eu.org>, 2003-2021.\\
	Kopiowanie, modyfikowanie i redystrybucja dozwolone pod warunkiem zachowania informacji o autorach.
}
\end{document}
