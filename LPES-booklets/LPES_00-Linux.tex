% Copyright (c) 2021 Matematyka dla Ciekawych Świata (http://ciekawi.icm.edu.pl/)
% Copyright (c) 2021 Robert Ryszard Paciorek <rrp@opcode.eu.org>
% 
% MIT License
% 
% Permission is hereby granted, free of charge, to any person obtaining a copy
% of this software and associated documentation files (the "Software"), to deal
% in the Software without restriction, including without limitation the rights
% to use, copy, modify, merge, publish, distribute, sublicense, and/or sell
% copies of the Software, and to permit persons to whom the Software is
% furnished to do so, subject to the following conditions:
% 
% The above copyright notice and this permission notice shall be included in all
% copies or substantial portions of the Software.
% 
% THE SOFTWARE IS PROVIDED "AS IS", WITHOUT WARRANTY OF ANY KIND, EXPRESS OR
% IMPLIED, INCLUDING BUT NOT LIMITED TO THE WARRANTIES OF MERCHANTABILITY,
% FITNESS FOR A PARTICULAR PURPOSE AND NONINFRINGEMENT. IN NO EVENT SHALL THE
% AUTHORS OR COPYRIGHT HOLDERS BE LIABLE FOR ANY CLAIM, DAMAGES OR OTHER
% LIABILITY, WHETHER IN AN ACTION OF CONTRACT, TORT OR OTHERWISE, ARISING FROM,
% OUT OF OR IN CONNECTION WITH THE SOFTWARE OR THE USE OR OTHER DEALINGS IN THE
% SOFTWARE.

\documentclass{pdfBooklets}

\title{Linux i Python w Elektronicznej Sieci:\\ Linux i Python w domu}
\author{%
	Projekt ,,Matematyka dla Ciekawych Świata'',\\
	Robert Ryszard Paciorek\\\normalsize\ttfamily <rrp@opcode.eu.org>
}
\date  {2021-03-01}

\renewcommand{\zaawansowane}[1]{%
	\ifnumcomp{#1}{<}{5}  {} {%
	\ifnumcomp{#1}{<}{15} {} {%
	\ifnumcomp{#1}{<}{25} {\Symbola 🤔} {\Symbola 🧐}%
	}}%
}

\makeatletter\hypersetup{
	pdftitle = {\@title}, pdfauthor = {\@author}
}\makeatother

\makeatletter\let\percentchar\@percentchar\makeatother
\newcommand{\draftDate}{
	\directlua{
		if not os.getenv("LPES_FINAL") then
			if os.getenv("LPES_DRAFT_DATE") then
				tex.sprint( " [draft " .. os.getenv("LPES_DRAFT_DATE") .. "]" )
			else
				tex.sprint( " [draft " .. os.date("\percentchar F") .. "]" )
			end
		end
	}
}
\makeatletter
\let\oldDate\@date
\date {\oldDate \color{red}{\textbf{\draftDate}}}
\makeatother

\newcommand{\baseURLtoLPES}{http://ciekawi.icm.edu.pl/lpes}


\NewDocumentCommand{\Zadania}{o m o}{
	\section{Zadania}
	\IfValueT{#1}{\input{#1}}
	\renewcommand*{\do}[1]{\input{##1}}
	\docsvlist{#2}
	
	\IfValueT{#3}{
		\vspace{1cm}
		\section{Zadania praktyczne}
		
		Zadania te, dokładnie w takiej samej formie, będziemy realizować wspólnie w ramach laboratorium, więc nie musisz ich robić samemu.
		Zamieszczamy je jednak z wyprzedzeniem, abyś wiedział(a) co cię czeka i upewnił(a) się że masz wszystkie potrzebne elementy pod ręką.
		
		\renewcommand*{\do}[1]{\dbEntryInsert{##1}{praktyczne}}
		\docsvlist{#3}
	}
	\renewcommand{\insertZadanie}[3]{\subsubsection*{Rozwiązanie zadania \ref{##2}} \dbEntryInsert{##1}{##2-rozwiazanie}}
	\newcommand{\noExtraInfoMode}{TRUE}
	\input{booklets-sections/rozwiazania-intro.tex}
	\begin{RozwiazanieBox}
	\renewcommand*{\do}[1]{\input{##1}}
	\docsvlist{#2}
	\end{RozwiazanieBox}
	\let\noExtraInfoMode\undefined
}

\usepackage{labels4easylist}

\begin{document}

\maketitle

W związku z on-line'ową formułą tegorocznej edycji będziemy od Was wymagać dostępu do komputera z systemem Linux.
W systemie powinny być zainstalowane podstawowe programy oraz:
\vspace{3pt}\begin{easylist}[itemize]
	& Python w wersji 3.x
	& bash
	& vim
	& gcc lub clang (wraz z wersją deweloperską biblioteki standardowej C)
	& g++ lub clang++ (wraz z wersją deweloperską biblioteki standardowej C++)
\end{easylist}

\noindent
Dodatkowo na części zajęć z elektroniki wymagany będzie dostęp do urządzeń USB w tym systemie oraz możliwość wcześniejszego doinstalowania kilku narzędzi i bibliotek.

\vspace{7pt}
Rekomendujemy aby był to \textbf{\textit{Debian GNU/Linux}} w aktualnej wersji \textit{stable}, czyli \textit{Buster},
	gdyż na takim systemie opracowywane były materiały do tego kursu i na nim też będzie nam najłatwiej udzielić pomocy i wsparcia przy ewentualnych problemach.
Może to być także inna dystrybucja Linuxa lub system z rodziny BSD,
a nawet \textit{macOS}\footnote{
	Należy jednak mieć na uwadze że \textit{macOS} jest trochę odmienny od innych systemów „unixowych”, w szczególności od systemów z rodziny Linux na których oparty jest ten kurs, więc mogą wystapić pewne problemy.
} lub \textit{Windows Subsystem for Linux}\footnote{
	Jako że jest to zasadniczo forma wirtualizacji to należy jednak mieć na uwadze że mogą wystąpić pewne problemy związane z dostępem do sprzętu potrzebnym w dalszej części zajęć.
}.


\section{Linux}

\vspace{7pt}
Istnieje kilka metod prowadzących do spełnienia wymogu posiadania systemu Linux.

\subsection{Osobny komputer z Linuxem}

Jeżeli masz monitor z wolnym wejściem HDMI lub DVI, stosowny kabel oraz wolną klawiaturę i myszkę USB wystarczy ...
	dokupić komputer jednopłytkowy typu Raspberry Pi z zasilaczem i kartą pamięci, aby mieć własny, osobny, niezależny komputer z Linuxem.
Zamiast Raspberry Pi możesz też kupić płytkę Banana Pi lub Orange Pi.
Warto zwrócić uwagę aby miała złącze HDMI oraz RJ45 (Ethernet).
Warto też zwrócić uwagę na porządny zasilacz i kabel zasilający (żeby nie był oszukany i dostatecznie dobrze przewodził prąd), bo większość problemów tych płytek wynika z zasilania.

\vspace{7pt}
System dla Raspberry możesz pobrać z \url{https://www.raspberrypi.org/software/operating-systems/#raspberry-pi-os-32-bit}, a dla pozostałych płytek z \url{https://www.armbian.com/download/}.
Wszystkie one są oparte na Debianie więc nie będzie problemu. System wystarczy nagrać na kartę SD włożyć ja do płytki *Pi i uruchomić.

\vspace{7pt}
Dodatkowo inwestując około 200 zł będziesz mieć też zabawkę która na pewno przyda się w dalszym rozwijaniu zainteresowań programistycznych i elektronicznych ... może warto?

\subsection{Instalacja na swoim komputerze}

System Linux może być zainstalowany "obok" innych systemów operacyjnych na jednym komputerze a nawet na jednym dysku twardym.
Instalację taka nazywa się \textit{dual boot}. Wymaga to wydzielenia dla niego przestrzeni na dysku w postaci osobnej partycji.
Możesz skorzystać na przykład z następujących poradników:
\vspace{3pt}\begin{easylist}[itemize]
	& Debian 10 (po angielsku) \url{https://www.hebergementwebs.com/gnu-linux/installing-debian-10-buster-in-dual-boot-with-windows-10-the-complete-guide}
	& Ubuntu 18.04 (video po polsku) \url{https://www.youtube.com/watch?v=Zcn4pnSFq2E}
	& Mint (video po polsku) \url{https://www.youtube.com/watch?v=fi-qy0vB9TA}
	& Mint (inne video po polsku) \url{https://www.youtube.com/watch?v=RlpDPcgM08k}
\end{easylist}

\vspace{7pt}
Oczywiście może być też zainstalowany zamiast innego systemu operacyjnego.
Jeżeli nie zależy nam na zachowaniu danych które mieliśmy na dysku to instalacja taka będzie bardzo prosta.
Możesz skorzystać z powyższych poradników po prostu nie przejmując się tworzeniem osobnej partycji i tym podobnymi działaniami mającymi na celu zachowanie zawartości dysku.

\subsection{Live USB}

Możesz utworzyć bootowalny pendrive i uruchamiać Linuxa z zewnętrznego nośnika USB.
Do utworzenia takiego pendrive można skorzystać np. z \url{https://www.pendrivelinux.com/universal-usb-installer-easy-as-1-2-3/}

\subsection{Wirtualizacja}

Możesz zainstalować także Linuxa jako maszynę wirtualną np. w przy pomocy \textit{virtualbox} – \url{https://www.virtualbox.org/} i \url{https://www.osboxes.org/virtualbox-images/}. Jednak polecamy raczej powyższe metody.


\section{Oprogramowanie}

W momencie gdy już masz Linuxa powinieneś doinstalować potrzebne oprogramowanie.
W systemie Debian (i~podobnych) w tym celu należy wykonać w terminalu polecenie:
\vspace{3pt}\begin{easylist}[itemize]
	& \Verb#su -c "apt install python3 vim gcc g++ clang" -#\\ (jeżeli w trakcie instalacji tworzyłeś/tworzyłaś hasło dla administratora / użytkownika root)
\end{easylist}
\vspace{3pt}\noindent  albo:
\vspace{3pt}\begin{easylist}[itemize]
	& \Verb#sudo -- apt install python3 vim gcc g++ clang#\\ (jeżeli konfigurowałaś/konfigurowałeś użytkownika z prawem do sudo)
\end{easylist}


\section{Pomoc}

Pierwsze spotkanie (2 marzec 2021) będzie miało charakter organizacyjny i będziesz mógł/mogła wtedy skonsultować z nami ewentualne problemy związane z instalacją systemu, jego konfiguracją lub instalacją oprogramowania.
W razie problemów lub pytań możesz też wysłać wcześniej e-mail na adres \url{ciekawi.pracownia@icm.edu.pl}. Chętnie pomożemy.


\copyrightFooter{
	© Matematyka dla Ciekawych Świata, 2021.\\
	© Robert Ryszard Paciorek <rrp@opcode.eu.org>, 2021.\\
	Kopiowanie, modyfikowanie i redystrybucja dozwolone pod warunkiem zachowania informacji o autorach.
}
\end{document}
